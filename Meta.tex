% !TEX encoding = UTF-8 Unicode
% !TEX root =  Arbeit.tex

% Meta-Informationen ------------------------------------------------------------------------------------
%   Definition von globalen Parametern, die im gesamten Dokument verwendet
%   werden können (z.B auf dem Deckblatt etc.).
%
%   ACHTUNG: Wenn die Texte Umlaute oder ein Esszet enthalten, muss der folgende
%            Befehl bereits an dieser Stelle aktiviert werden:
%            \usepackage[latin1]{inputenc}
% -------------------------------------------------------------------------------------------------------
\newcommand{\titel}{Titel der Arbeit}
\newcommand{\titelDeckblatt}{Titel der Arbeit}

\newcommand{\untertitel}{Untertitel der Arbeit}
\newcommand{\untertitelDeckblatt}{Untertitel der Arbeit\\ f\"ur das Deckblatt}

\newcommand{\art}{Bachelor-Thesis}
\newcommand{\fachgebiet}{zur Erlangung des akademischen Grades\\ Bachelor of Science (B.\,Sc.) im Studienfach\xspace}
\newcommand{\studienbereich}{Angewandte Informatik}

\newcommand{\autor}{Moritz Rupp}

\newcommand{\keywords}{Titel der Arbeit, Fachgebiet, Forschung}

\newcommand{\matrikelnr}{1234567}
\newcommand{\kurs}{KursABC1}

\newcommand{\erstgutachter}{Prof. Dr. Max Mustermann}
\newcommand{\zweitgutachter}{Dipl.-Ing. (FH) Herbert Beispiel}

\newcommand{\jahr}{2013}

\newcommand{\hochschule}{Duale Hochschule Baden-W\"urttemberg}
\newcommand{\firma}{Musterfirma GmbH}
\newcommand{\ort}{Stuttgart}

\newcommand{\logo}{LogoMuster.pdf}
\newcommand{\creator}{texmaker 4.0.2}

\newcommand{\bibliographyheading}{Literaturverzeichnis}

\newcommand{\TableCellPadding}{1.2}