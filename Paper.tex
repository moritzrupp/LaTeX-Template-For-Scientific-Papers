% !TEX encoding = UTF-8 Unicode

% ------------------------------------------------------------------------------------------------------
% Style template for scientific papers (such as bachelor or master thesis, seminar papers, documentations)
% ------------------------------------------------------------------------------------------------------
%	originally created by Stefan Macke, 24/04/2009
%	http://blog.stefan-macke.de
%
%	for the first time expanded by Felix Rupp, 21/05/2009
%	http://www.felixrupp.com/
%
%
%
% expaneded and customised by Moritz Rupp
% moritz.rupp@gmail.com
% http://www.moritzrupp.de
%
% Version 2.0.1
% Date: 26/01/2014


% Document head ---------------------------------------------------------------------------------------
% This template is based on 'scrreprt' of the koma-script.
% ------------------------------------------------------------------------------------------------------
\documentclass[
    12pt, % font-size
    DIV10, % Change of the size of the print space, only to be used with koma-script
    UKenglish, % langauge
    a4paper, % paper size
    oneside, % one or twoside
    titlepage, % titlepage is used
    parskip=half, % Space between paragrpahs (half of a line)
    headings=normal, % Reduce size of headings
    listof=totoc, % list the listings in the table of contents
    bibliography=totoc, % list the bibliography in the table of contents
    index=totoc, % list the index in the table of contents
    captions=tableheading, % captions of tables etc. below
    final % status of the document (final/draft)
]{scrreprt}

% UTF8 and T1 Fontencoding -----------------------------------------------------------------------------
\usepackage[utf8]{inputenc}
\usepackage[T1]{fontenc}


% Meta information -----------------------------------------------------------------------------------
% Information about the document, such as title, author, student id, etc
% are defined in the file Meta.tex. They can be used globally.
% ------------------------------------------------------------------------------------------------------
% !TEX encoding = UTF-8 Unicode
% !TEX root =  Paper.tex

% Meta information ------------------------------------------------------------------------------------
% Definiton of global parameters, they can be used in the whole document (e.g. on the titlepage)
% -------------------------------------------------------------------------------------------------------
\newcommand{\myTitle}{Title of the paper}
\newcommand{\mySubtitle}{the subtitle}
\newcommand{\mySubtitleTitlepage}{The subtitle}
\newcommand{\kind}{Bachelor Thesis}
\newcommand{\area}{Submitted in partial fulfilment of the requirements for the degree of Bachelor of Science (B.\,Sc.) in the degree programme \xspace}
\newcommand{\myAuthor}{Moritz Rupp}
\newcommand{\keywords}{Bachelor thesis, Moritz Rupp}
\newcommand{\study}{Applied Computer Science}
\newcommand{\studentno}{1234567}
\newcommand{\course}{KursABC1}
\newcommand{\firstreviewer}{Prof. Dr. Max Mustermann}
\newcommand{\secondreviewer}{Dipl.-Ing. (FH) Herbert Beispiel}
\newcommand{\myYear}{2013}
\newcommand{\university}{Baden-Wuerttemberg Cooperative State University}
\newcommand{\company}{Musterfirma GmbH}
\newcommand{\place}{Stuttgart}
\newcommand{\logo}{LogoMuster.pdf}
\newcommand{\creator}{texmaker 4.0.1}
\newcommand{\bibliographyheading}{Bibliography}


% necessary packages -----------------------------------------------------------------------------------
% Necessary LaTeX packages  are included in the external file Packages.tex
% to keep the tepmlate clean.
% ------------------------------------------------------------------------------------------------------
% !TEX encoding = UTF-8 Unicode
% !TEX root =  Paper.tex

% Customisation of the page layout ---------------------------------------------------
%   see Layout.tex
% -----------------------------------------------------------------------------------------
\usepackage[
    automark, % automatically create chapter name in the headline
    headsepline, % seperation line beneath headline
    ilines % left seperate line
]{scrpage2}


% Customisation of the language -------------------------------------------------
\usepackage[UKenglish]{babel}


% Special character 
% -----------------------------------------------------------------------------------------
\usepackage{textcomp} % Euro-Zeichen etc.


% Font --------------------------------------------------------------------------------
\usepackage{lmodern} % better fonts
\usepackage{relsize} % font size relatively


% Better underlining ---------------------------------------------
\usepackage[normalem]{ulem}


% Graphics -----------------------------------------------------------------------------
% Enable the inclusion of JPEG
\usepackage[dvips,final]{graphicx}
% Path to the graphics
\graphicspath{{Images/}}
\usepackage{picture}


% Commands from AMSTeX for mathematical symbols, e.g.  \boldsymbol \mathbb
\usepackage{amsmath,amsfonts}


% Use Euro sign ------------------------------------------------------------
\usepackage{eurosym}


% For index with \printindex -----------------------------------------------
\usepackage{makeidx}


% Easy definition of margins and line spacing etc. ------------
\usepackage{setspace}
\usepackage{geometry}


% For the list of abbreviations ------------------------
\usepackage[
	printonlyused,
	footnote
]{acronym}

% For the glossary -------------------------
\usepackage[
	nonumberlist, %keine Seitenzahlen anzeigen
	%acronym,      %ein Abkürzungsverzeichnis erstellen
	%section,     %im Inhaltsverzeichnis auf section-Ebene erscheinen
	toc,          %Einträge im Inhaltsverzeichnis
]{glossaries}

% Floating figures ---------------------------------------------------------
\usepackage[vflt]{floatflt}
\usepackage{subfigure}

% Inclusion of program code -----------------------------------------------
\usepackage{listings}
\usepackage{xcolor} 
\definecolor{hellgelb}{rgb}{1,1,0.9}
\definecolor{colKeys}{rgb}{0.8,0,0.5}
\definecolor{colIdentifier}{rgb}{0.6,0,0.3}
\definecolor{colComments}{rgb}{0,0.5,0}
\definecolor{colString}{rgb}{0,0,1}

\lstset{
    float=htbp,
    basicstyle=\ttfamily\color{black}\small\smaller,
    identifierstyle=,%\color{colIdentifier},
    keywordstyle=\color{colKeys}\bfseries,
    stringstyle=\color{colString},
    commentstyle=\color{colComments},
    columns=flexible,
    tabsize=4,
    frame=single,
    extendedchars=true,
    showspaces=false,
    showstringspaces=false,
    numbers=left,
    numberstyle=\tiny,
    breaklines=true,
    backgroundcolor=\color{hellgelb},
    breakautoindent=true,
    escapeinside={(*}{*)},
    literate={Ö}{{\"O}}1 {Ä}{{\"A}}1 {Ü}{{\"U}}1 {ß}{{\ss}}2 {ü}{{\"u}}1 {ä}{{\"a}}1 {ö}{{\"o}}1 {µ}{\textmu}1
 }

% --------------------------------------------------------------------------------------------
% 
% Definition of source code
%
% Define CSS:
\lstdefinelanguage{CSS}{
    morestring=[b]',
    morestring=[b]",
    comment=[l]{/*}{*/},
    sensitive=false,
    morekeywords={accelerator,adjust,after,align,attachment,azimuth,background,before,behavior,binding,border,bottom,bottomright,bottomleft,break,caption-side,char,clear,clip,color,colors,cue,cursor,collapse,decoration,direction,display,elevation,empty-cells,family,filter,float,flow,focus,font,grid,height,image,increment,indent,input,inside,ime-mode,include-source,justify,last,layer,left,letter,line,list,margin,marker,marks,max,min,mode,modify,moz,offsett,opacity,orphans,outline,overflow,overflow-X,overflow-Y,overhang,position,position-x,position-y,padding,page,pause,pitch,play-during,position,quotes,radius,range,repeat,replace,reset,richness,right,ruby,set-link-source,select,shadow,size,spacing,speak,speak-header,speak-numeral,speak-punctuation,speech-rate,stress,stretch,style,table-layout,text,transform,text-autospace,text-kashida-space,top,topleft,topright,type,underline,unicode-bidi,use-link-source,user,variant,vertical,visibility,voice-family,volume,white-space,weight,widows,width,word,wrap,word-wrap,writing-mode,z-index,zoom}
}


% Define JavaScript:
\lstdefinelanguage{JavaScript}{
  keywords={typeof, new, true, false, catch, function, return, null, catch, switch, var, if, in, while, do, else, case, break},
  keywordstyle=\color{colKeys}\bfseries,
  ndkeywords={class, export, boolean, throw, implements, import, this},
  ndkeywordstyle=\color{darkgray}\bfseries,
  sensitive=false,
  comment=[l]{//},
  morecomment=[s]{/*}{*/},
  morestring=[b]',
  morestring=[b]"
}

% Define TypoScript:
\lstdefinelanguage{TypoScript}{
  keywords=[1]{PAGE, HTML, TEXT, COA, COA\_INT, FILE, IMAGE, IMG\_RESOURCE, CLEARGIF, CONTENT, RECORDS, HMENU, CTABLE, OTABLE, COLUMNS, HRULER, IMGTEXT, CASE, LOAD\_REGISTER, RESTORE\_REGISTER, FORM, SEARCHRESULT, USER, USER\_INT, TEMPLATE, FLUIDTEMPLATE, MULTIMEDIA, SVG, EDITPANEL, GIFBUILDER, GMENU, TMENU, TMENUITEM, IMGMENU, IMGMENUITEM, JSMENU, JSMENUITEM, BOX},
  keywordstyle=[1]{\color{blue}\bfseries},
  keywords=[2]{plugin, view, page, file, text, config},
  keywordstyle=[2]{\color{blue}\bfseries},
  keywords=[3]{EXT},
  keywordstyle=[3]{\color{blue}\bfseries},
  sensitive=true,
  comment=[l]{\#\ }
}

% PHP5 dialect 
\lstdefinelanguage[5]{PHP}[]{PHP} {
  morekeywords={
  %--- class and exceptions keywords 
  class,static,private,public,abstract,interface,const,function,require_once,final,new,extends,implements,
  %--- additional array functions 
  array_combine,array_diff_uassoc,array_udiff,array_udiff_assoc,% 
  array_udiff_uassoc,array_walk_recursive,array_uintersect_assoc,% 
  array_usintersect_uassoc,array_uintersect,% 
  %--- string functions 
  str_split, strpbrk,substr_compare 
  %--- Date and time functions 
  idate,date_sunset,date_sunrise,time_nanosleep,
  %--- return
  return
  } 
} 


% Fluid dialect 
\lstdefinelanguage[Fluid]{HTML}[]{HTML} {
  morekeywords={if, for, each, as, condition, controller, arguments, image, link, action, class} 
} 


% link URL, break long URL etc. -------------------------------------
\usepackage{url}


% include biblatex -----------------------------------------------------------------------
\usepackage[style=authoryear-ibid, hyperref=true]{biblatex}


% PDF Options -------------------------------------------------------------------------
\usepackage[
    bookmarks,
    bookmarksopen=true,
    colorlinks=true,
% use this colour definitions for links with colour
    linkcolor=red, % simple internal link
    anchorcolor=black,% anchor text
    citecolor=blue, % Anchor to bibliography entries in the text
    filecolor=magenta, % Anchors to local files
    menucolor=red, % Acrobat menu entries
    urlcolor=cyan, 
%
% use this colour definitions for pring: everything is black
%
    %linkcolor=black, % simple internal link
    %anchorcolor=black, % anchor text
    %citecolor=black, %Anchor to bibliography entries in the text
    %filecolor=black, % Anchors to local files
    %menucolor=black, % Acrobat menu entries
    %urlcolor=black, 
%
%    backref,
    plainpages=false, % for the correct creation of bookmarks
    pdfpagelabels, % for the correct creation of bookmarks
    hypertexnames=false, % for the correct creation of bookmarks
    linktocpage % link the page numbers instead of the whole toc entry
]{hyperref}
% Commands with print special characters cause errors if they are used with hyperref
\hypersetup{
    pdftitle={\myTitle \mySubtitle},
    pdfauthor={\myAuthor},
    pdfcreator={\creator},
    pdfsubject={\myTitle \mySubtitle},
    pdfkeywords={\keywords},
}

% Package for a clean inclusion of external PDF files -----------------
\usepackage[final]{pdfpages}


% Continuous numeration of footnotes -------------------------------
\usepackage{chngcntr}


% nicer tables --------------------------------------------------------------------
\usepackage{tabularx}
\usepackage{multirow}


% for long tables ---------------------------------------------------------------------
\usepackage{longtable}
\usepackage{ltxtable}
\usepackage{filecontents}
\usepackage{array}
\usepackage{ragged2e}
\usepackage{lscape}


% rotation of elements -------------------------------------------------------
\usepackage{rotating}


% Change the formatting of lists --------------------------------------------------
\usepackage{paralist}
\usepackage{enumitem}


% Necessary for the creation of own commands -------------------------------------
\usepackage{ifthen}
\usepackage{forloop}


% defines commands for \todo and \listoftodos --------------------------------
\usepackage[english]{todonotes}


% used for the end of macros without parameter
\usepackage{xspace}



% Activate the creation of an index, list of abbreviations and glossary ------------------------------
\makeindex{}


% Head- and footline, margins etc. ---------------------------------------------------------------
% !TEX encoding = UTF-8 Unicode
% !TEX root =  Paper.tex

% Line spacing 1,5 lines ---------------------------------------------------------------------
\onehalfspacing{}


% Margins ------------------------------------------------------------------------------------
\setlength{\topskip}{\ht\strutbox} % behebt Warnung von geometry
\geometry{paper=a4paper,left=35mm,right=35mm,top=35mm}


% Head- and footline --------------------------------------------------------------------------
\pagestyle{scrheadings}

% pring head- and footline at the beginning of a chapter
\renewcommand*{\chapterpagestyle}{scrheadings} 

% font style of the headline
\renewcommand{\headfont}{\normalfont}


% Headline for onesided documents:
\ihead{
	\textit{\headmark}
}
\chead{}
\rohead{\includegraphics[scale=0.06]{\logo}}
\setlength{\headheight}{21mm} % Height of the headline

% expand the headline further the text
%\setheadwidth[0pt]{textwithmarginpar} 
\setheadsepline[text]{0.4pt} % seperation line beneath the headline

%% headline for twosided documents:
%\ihead{\textit{\leftmark}}
%\chead{}
%\ohead{}
%\lehead{\includegraphics[scale=0.25]{\logo}}
%\rohead{\includegraphics[scale=0.25]{\logo}}
%\setheadsepline[text]{0.4pt} % seperation line beneath the headline


% Footline
%\cfoot{-- DRAFT -- \\[4ex]} % uncomment during draft status
\cfoot{} % comment during draft status
\ofoot{\pagemark \\[4ex]}


% 	miscellaneous settings ---------------------------------------------------
% a little bit more spacing after a fullstop
\frenchspacing{}

% avoide orphans and widow lines
\clubpenalty = 10000
\widowpenalty = 10000 
\displaywidowpenalty = 10000

% format source code
\lstset{numbers=left, numberstyle=\tiny, numbersep=5pt, breaklines=true}
\lstset{emph={square}, emphstyle=\color{red}, emph={[2]root,base}, emphstyle={[2]\color{blue}}}

% footnotes are counted continuously
\counterwithout{footnote}{chapter}

% more space in tables
\renewcommand{\arraystretch}{\TableCellPadding}

% Layout for bibliography
\setlength\bibitemsep{2\itemsep} % Set the space between bibliography entries
\setlength{\bibhang}{0.2cm}

% Bold bibliography entries
\xpretobibmacro{author}{\mkbibbold\bgroup}{}{}
\xapptobibmacro{author}{\egroup}{}{}
\xpretobibmacro{bbx:editor}{\mkbibbold\bgroup}{}{}
\xapptobibmacro{bbx:editor}{\egroup}{}{}
\renewcommand*{\labelnamepunct}{\mkbibbold{\addcolon\space}}


% Own definitions for hyphenation ---------------------------------------------------------------
% !TEX encoding = UTF-8 Unicode
% !TEX root =  Paper.tex

% Define custom hyphenation with \hyphenation.



% Own and custom LaTeX commands ---------------------------------------------------------------------------------
% !TEX encoding = UTF-8 Unicode
% !TEX root =  Paper.tex
% Custom commands


% simple changing of the font, e.g.: \changefont{cmss}{sbc}{n} ---------------------------------------
\newcommand{\changefont}[3]{\fontfamily{#1} \fontseries{#2} \fontshape{#3} \selectfont}

\newcommand{\bs}{$\backslash$}


% List elements with bold heading ------------------------------------------------------
\newcommand{\itemd}[2]{\item{\textbf{#1}}\\{#2}}


% print authors
\newcommand{\AuthorName}[1]{\textsc{#1}}
\newcommand{\Author}[1]{\AuthorName{\citeauthor{#1}}}


% commands for highlighting words -----------------------------------------------
\newcommand{\NewTerm}[1]{\textit{#1}}


% Amount with currency -----------------------------------------------------------------------------------
\newcommand{\Amount}[2][general]{#2\,\ifthenelse{\equal{#1}{dollar}}{\$}{}\ifthenelse{\equal{#1}{euro}}{€}{}\ifthenelse{\equal{#1}{yen}}{¥}{}\ifthenelse{\equal{#1}{cent}}{¢}{}\ifthenelse{\equal{#1}{pound}}{£}{}\ifthenelse{\equal{#1}{peso}}{₱}{}\ifthenelse{\equal{#1}{baht}}{฿}{}\ifthenelse{\equal{#1}{franc}}{₣}{}\ifthenelse{\equal{#1}{lira}}{₤}{}\ifthenelse{\equal{#1}{drachma}}{₯}{}\ifthenelse{\equal{#1}{pfennig}}{₰}{}\ifthenelse{\equal{#1}{general}}{¤}{}}


% Miscellaneous (samples)---------------------------------------------------------------------------------------------
\newcommand{\Input}[1]{\texttt{#1}}
\newcommand{\Code}[1]{\texttt{#1}}
\newcommand{\File}[1]{\texttt{#1}}


% Column defintion right-aligned with defined width -------------------------------------------------
\newcolumntype{w}[1]{>{\raggedleft\hspace{0pt}}p{#1}}

% Left-aligned table columns with tabularx -------------------------------------------------------------
\newcolumntype{y}[1]{>{\RaggedRight\arraybackslash\hsize=#1\hsize}X}



% Bibliography ---------------------------------------------------------------------------------------
% The bibliogrphay is generated from the BibTeX database "Bibliography.bib" 
% ------------------------------------------------------------------------------------------------------
\bibliography{Bibliography} % Call: biber Bibliography

% For different sections in the bibliography use the defbibheading command -----------------------------------------------
\defbibheading{internet}{\section*{Internet and Intranet Resources}}
\defbibheading{literature}{\section*{Literature}}

% Glossary --------------------------------------------------------------------------------
\makeglossaries
\loadglsentries{Content/Glossary.tex}

% Add the unused glossary entries (in case you have some you only use as acronyms)
%\glsadd{sample}

% The actual document
%  -----------------------------------------------------------------------------
% The actual content of the document is beginning here. 
% The seperate pages and chapter are sourced out and included here.
% ------------------------------------------------------------------------------------------------------
\begin{document}


% give numbers to subsubsections ----------------------------------------------------------------------
\setcounter{secnumdepth}{3}
% Depth of the table of contents (don't show subsubsections
\setcounter{tocdepth}{2}


% Titlepage and abstract wihtout page numbering ---------------------------------------------------------------
\ofoot{}
% !TEX encoding = UTF-8 Unicode
% !TEX root =  Paper.tex

\newgeometry{margin=30mm}

\thispagestyle{plain}
\begin{titlepage}

%\changefont{cmss}{bx}{n}

\begin{center}
{\bfseries

\Large{\uline{\myTitleTitlepage}}\\[16ex]

{\large
Degree Dissertation\\
for the\\
Master Examination in \study\\
at the\\
\area\\
of the \\
\university\\
\place
}}
\end{center}

\vspace{\fill}

\begin{flushright}
\renewcommand{\arraystretch}{1}
\begin{tabular}{ll}
& \quad Examiner:\\
& \quad \firstreviewer\\
& \quad \\
& \quad Submitted by:\\
& \quad \myAuthor\\
& \quad Born in \born\\
& \quad \\
Date of submission: & \quad \submissionDate
\end{tabular}
\renewcommand{\arraystretch}{\TableCellPadding}
\end{flushright}

\end{titlepage}

\restoregeometry


% Decleration of Honour ----------------------------------------------------------------------------
% !TEX encoding = UTF-8 Unicode
% !TEX root =  Paper.tex

\chapter*{Declaration of Honour}

\thispagestyle{empty}

I, \myAuthor, student ID \studentID, hereby declare that this \kind\xspace with the title

\begin{quote}
\textit{\myTitle. \mySubtitle}
\end{quote}

has been written only by the undersigned and without any assistance from third parties. Furthermore, I confirm that no sources have been used in the preparation of this paper other than those indicated in the paper itself. The paper has not been submitted to any university or institution before.

\vspace{8ex}

\place, 4th October 2013

\vspace{6ex}

\rule[-0.2cm]{5cm}{0.5pt}

\textsc{\myAuthor} 

% Non-disclosure Note ----------------------------------------------------------------------------
% !TEX encoding = UTF-8 Unicode
% !TEX root =  Paper.tex

\chapter*{Non-Disclosure Note}

\thispagestyle{empty}

This {\kind} with the title ``\textit{\myTitle. \mySubtitle}'' may contain internal and confidential information of the \company{}. Every inspection, publication or copying -- even in parts of the paper or in digital form -- is generally prohibited. Exceptions need the written permission of the \company{}.

It is only presented to the board of examiners of the course of studies \study{} at the \university{} \place{}.


%  Abstract ------------------------------------------------------------------------------------------
% !TEX encoding = UTF-8 Unicode
% !TEX root =  ../Paper.tex

\chapter*{Abstract}
\label{cha:Abstract}

\thispagestyle{empty}


\blindtext[2]

\vspace{\fill}

\begin{description}
	\item[Keywords:] \keywords
\end{description}

% List of todos ---------------------------------------------------------------------------------------
% Uncomment for printing the list of todos.
%\phantomsection
%\markboth[Todo list}{Todo list}
%\listoftodos
%\newpage


% Page numbering ------------------------------------------------------------------------------------
%   Capital roman numbers before the main part.
% ------------------------------------------------------------------------------------------------------
\ofoot{\pagemark \\[4ex]}
\pagenumbering{Roman}

% Table of contents --------------------------------------------------------------------------------------
\phantomsection{} % Guarantees the correct inclusion of the section into the table of contents
\addcontentsline{toc}{chapter}{Contents}
\tableofcontents{}


%%
% Print abbreviations and glossary ------------------------------------------------------
%%

% print acronyms
\newpage
\renewcommand{\glsnamefont}[1]{\makefirstuc{#1}} %make the first letter uppercase in the glossary
\printglossary[style=altlist,type=main]
\label{cha:Glossary}

% print acronyms
\newpage
\markboth{List of Abbreviations}{List of Abbreviations}
\printglossary[style=listdotted, type=\acronymtype, title={List of Abbreviations}]
\label{cha:Abbreviations}

% Remaining listings ----------------------------------------------------------------------------------
\listoffigures{} % List of figures
\listoftables{} % List of tables
\renewcommand{\lstlistlistingname}{List of Listings}
\lstlistoflistings{} % List of listings

% arabic page numbering in the main part ------------------------------------------------------------------
\clearpage{}
\pagenumbering{arabic}


% the conent chapters are included in "Content.tex"  --------------------------------------------------------
% !TEX encoding = UTF-8 Unicode
% !TEX root =  Paper.tex

% Here you can include the different chapters. 
% They have to be in the corresponding .TEX file.
% If you want you can change the file names.


% Put all your chapter/section related text HERE (just copy paste)

\chapter{Introduction}
\label{cha:Introduction}

This is a simple introduction to the template. It shows several functions supported by the template like a glossary, acronyms, citing, referencing, tables and figures (see~\cref{sec:TablesFigures}), lists, listings, etc.

\section{Example of the Glossary}
\label{sec:ExGlossary}

This work is published under the \gls{CC-SA} license. So, feel free to use it for your thesis and adapt it to your needs. Because of the \ac{CC} license your allowed to do so. But first of all you should learn {\LaTeX}. Some people think its difficult to learn, because there is no \ac{WYSIWYG} editor.


\section{Citing}
\label{sec:Citing}

This template uses the Harvard citing style \enquote{author-year}. But it's to change, e.g. if you don't like in line citing with parentheses just start using footnotes. Here is an example; a wise quote by \textcite{Asimov1997}:

\begin{quote}
\enquote{Knowledge is indivisible. When people grow wise in one direction, they are sure to make it easier for themselves to grow wise in other directions as well. On the other hand, when they split up knowledge, concentrate on their own field, and scorn and ignore other fields, they grow less wise \textemdash\ even in their own field.}
\end{quote}

As you have guessed, this template doesn't use the default {\LaTeX} fonts. In order to do so, the package \emph{fontspec} is included. With \emph{fontspec} the user is able to load OpenType fonts in a {\LaTeX} document \autocite[see][3]{Robertson2013}. But in this case, \emph{Times New Roman} is chosen which is supported by {\LaTeX}.

\section{Tables and Figures}
\label{sec:TablesFigures}

This section has examples of tables (see~\cref{tab:SampleTable}) and figures (see~\cref{fig:PlaceHolder}).

\begin{figure}%
\caption{A nice and easy to use place holder.}%
\missingfigure{This is a place holder for a figure.}%
\caption*{Source: The reference is below the figure}
\label{fig:PlaceHolder}%
\end{figure}

\begin{center}
\begin{longtable}{c|c|c|c|c}
\caption{A sample table with performance data.}
\label{tab:SampleTable} \\

\textbf{\#}	&	\textbf{C} {[}sec{]}	&	\textbf{Java} {[}sec{]}	&	\textbf{Java - C} {[}sec{]}	&	\%	\\
\toprule[1.5pt]
\endfirsthead

\multicolumn{5}{c}{\tablename\ \thetable\ -- \textit{Continued from previous page}} \\
\textbf{\#}	&	\textbf{C} {[}sec{]}	&	\textbf{Java} {[}sec{]}	&	\textbf{Java - C} {[}sec{]}	&	\%	\\
\toprule[1.5pt]
\endhead
			
\multicolumn{5}{r}{\footnotesize \textit{Continued on next page}} \\
\endfoot
\endlastfoot

1	&	2.47	&	3.336932371	&	0.866932371	&	35.09847656	\\
2	&	2.49	&	3.28858181	&	0.79858181	&	32.07155863	\\
3	&	2.48	&	3.29198301	&	0.81198301	&	32.7412504	\\
4	&	2.49	&	3.286295345	&	0.796295345	&	31.97973273	\\
5	&	2.49	&	3.288482704	&	0.798482704	&	32.06757847	\\
6	&	2.48	&	3.290081162	&	0.810081162	&	32.66456298	\\
7	&	2.48	&	3.285896621	&	0.805896621	&	32.49583149	\\
8	&	2.49	&	3.289358723	&	0.799358723	&	32.10275996	\\
9	&	2.48	&	3.288659403	&	0.808659403	&	32.60723399	\\
10	&	2.49	&	3.335017952	&	0.845017952	&	33.93646394	\\
11	&	2.48	&	3.287597187	&	0.807597187	&	32.5644027	\\
12	&	2.49	&	3.293863277	&	0.803863277	&	32.28366574	\\
13	&	2.48	&	3.291244927	&	0.811244927	&	32.71148899	\\
14	&	2.48	&	3.286065776	&	0.806065776	&	32.50265226	\\
15	&	2.49	&	3.288609327	&	0.798609327	&	32.07266373	\\
16	&	2.49	&	3.288866762	&	0.798866762	&	32.08300249	\\
17	&	2.48	&	3.292376565	&	0.812376565	&	32.75711956	\\
18	&	2.49	&	3.290268407	&	0.800268407	&	32.13929345	\\
19	&	2.48	&	3.289624469	&	0.809624469	&	32.64614794	\\
20	&	2.48	&	3.288649694	&	0.808649694	&	32.6068425	\\
21	&	2.49	&	3.293171429	&	0.803171429	&	32.25588068	\\
22	&	2.48	&	3.291608451	&	0.811608451	&	32.72614722	\\
23	&	2.48	&	3.282929763	&	0.802929763	&	32.37620012	\\
24	&	2.49	&	3.293829963	&	0.803829963	&	32.28232783	\\
25	&	2.48	&	3.285785784	&	0.805785784	&	32.49136226	\\
\multicolumn{5}{c}{\ldots} \\
\bottomrule[2pt]
AVG:	&	2.4838	&	3.290900608	&	0.807100608	&	32.49540061	\\
STD	&	0.005674864	&	0.009707116	&	0.011960919	&	0.530352252	\\
\caption*{Source: The source of this table.}
\end{longtable}
\end{center}

\section{Lists}
\label{sec:Lists}

Lists look like this:

\blinditemize[5]


\chapter{How To}
\label{cha:HowTo}

This chapter is a guide on how to use this template. For example, the referencing and citing is explained. Moreover, the usage of figures and tables is elaborated.

\section{Build the document}
\label{sec:BuildDocument}

In order to build your document, you just have to double-click the file \File{makefile.bat}.

\section{Use of the Glossary}
\label{sec:UseGlossary}

In order to use the glossary, an entry has to be put in the file \File{Content/Glossary.tex}. \Cref{lst:glossary} shows a sample entry and the command for using it in the text. In the Glossary, a dot is automatically added at the end of each description.\footnote{This is an example of a footnote. Each page starts by 1 (can be changed for continuous numbering.}

\begin{lstlisting}[language={[LaTeX]TeX},label={lst:glossary},caption={The use of the Glossary.}]
\newglossaryentry{CC-SA}{
	name={Creative Commons Attribution-ShareAlike 3.0 Unported},
	description={This \acf*{CC} license allows the user to \emph{share} (copy and redistribute) a work and to \emph{adapt} (remix, transform and build upon) the work for \emph{any} purpose \autocite{creativecommons2014}}
}

@\ldots@ \gls{CC-SA} is used @\ldots@
\end{lstlisting}

\section{Use of Abbreviations/Acronyms}
\label{sec:UseAcronyms}

Acronyms/Abbreviations are also added to the file \File{Content/Glossary.tex}. \Cref{lst:acronyms} shows a sample acronym entry and the usage of it. There are many commands to use acronyms:

\begin{itemize}
	\item \lstinline[language={[LaTeX]TeX}]!\ac{<label>}! is the default command. First, the whole word is written. Afterwards only the short form is used
	\item \lstinline[language={[LaTeX]TeX}]!\acl{<label>}! always uses the long form
	\item \lstinline[language={[LaTeX]TeX}]!\acs{<label>}! is the short form
	\item \lstinline[language={[LaTeX]TeX}]!\acf{<label>}! writes the full acronym: Longform (shortform)
	\item Each command can be used with a \lstinline!*!, so that the produced text is not referenced\footnote{Another random footnote without important content.}
\end{itemize}

\begin{lstlisting}[language={[LaTeX]TeX},label={lst:acronyms},caption={The use of Abbreviations/Acronyms.}]
\newacronym{CC}{CC}{Creative Commons}
\newacronym{WYSIWYG}{WYSIWYG}{What You See Is What You Get}

\newacronym{<label>}{Shortform}{Longform}
\end{lstlisting}

\section{Referencing Commands}
\label{sec:ReferencingCommands}

Each figure, formulae or table can be accessed by a label for referencing. Therefore, two important commands are available:

\begin{enumerate}
	\item \lstinline[language={[LaTeX]TeX}]!\Cref{<label>}!
	\item \lstinline[language={[LaTeX]TeX}]!\cref{<label>}!
\end{enumerate}

The first capitalizes the first letter of \enquote{Figure}, \enquote{Tables}, etc. \Cref{lst:referencing} shows these two sentences in {\LaTeX} code where \texttt{lst:referencing} is the label of the listing.

\begin{lstlisting}[language={[LaTeX]TeX},label={lst:referencing},caption={How to reference.}]
The first capitalizes the first letter of \enquote{Figure}, \enquote{Tables}, etc. \Cref{lst:referencing} shows these two sentences in {\LaTeX} code where \texttt{lst:referencing} is the label of the listing.
\end{lstlisting}

\section{Quoting}
\label{sec:Quoting}

Quoting using this {\LaTeX} template is highly automated. There are several commands introduced in this section.

\begin{itemize}
	\item \lstinline[language={[LaTeX]TeX}]!\enquote{Text}!: Puts the text in quotation marks
	\item \lstinline[language={[LaTeX]TeX}]!\enquoteit{Text}!: Puts the text in quotation marks and italicizes the text (given by the Guidelines)
	\item \lstinline[language={[LaTeX]TeX}]!\foreignquote{<lang>}{Text}!: Should be used for quotes in foreign language, e.\,g.\,German.
	\item \lstinline[language={[LaTeX]TeX}]!\foreignquoteit{<lang>}{Text}!: See \lstinline[language={[LaTeX]TeX}]!\enquoteit{Text}!
	\item \lstinline[language={[LaTeX]TeX}]!\textquote[<cite>][<punct>]{<text>}<tpunct>!: This command can be used for direct quotations. It puts the \texttt{<text>} in quotation marks followed by a reference to the author. For \texttt{<cite>} you use the regular \lstinline[language={[LaTeX]TeX}]!\cite[<e.g. see>][<page>]{<citekey>}! command (in curly braces, not autocite command needed, because parenthesis are added automatically). \texttt{<punct>} is the punctuation after the text. \texttt{<tpunct>} is a punctuation after the whole command
\end{itemize}

The commands \lstinline[language={[LaTeX]TeX}]!\autocite[<e.g. see>][<page>]{<citekey>}! and \lstinline[language={[LaTeX]TeX}]!\textcite[<e.g. see>][<page>]{<citekey>}! are used for indirect quotes as usual.

\subsection{Examples}
\label{sec:QuotingExamples}

\begin{itemize}
	\item \lstinline[language={[LaTeX]TeX}]!\enquote{Text}!: \enquote{This text is in quotation marks}
	\item \lstinline[language={[LaTeX]TeX}]!\enquoteit{Text}!: \enquoteit{This text is in quotation marks and italicized}
	\item \lstinline[language={[LaTeX]TeX}]!\foreignquote{<lang>}{Text}!: \foreignquote{ngerman}{Dieser Text ist in Deutsch}
	\item \lstinline[language={[LaTeX]TeX}]!\foreignquoteit{<lang>}{Text}!: \foreignquoteit{ngerman}{Dieser Text ist in Deutsch und kursiv}
	\item \lstinline[language={[LaTeX]TeX}]!\textquote[<cite>][<punct>]{<text>}<tpunct>!: \textquote[{\cite[see][3]{Robertson2013}}]{Without fontspec, it is necessary to write cumbersome font definition files for {\LaTeX}, since \LaTeX{'s} font selection scheme (known as the \enquote{\textsc{nfss}} has a lot going on behind the scenes to allow easy commands like \texttt{\bs emph} or \texttt{\bs bfseries}}
	\item Multiple citations from the same author in a row are shorted: \cite[see][3]{Robertson2013}
	\item But only if they are on the same page: \cite[see][3]{Robertson2013}
\end{itemize}


\chapter{Some Blindtext}
\label{cha:SomeBlindtext}

\blindtext

\section{Some Paragraphs}
\label{sec:SomeParagraphs}

\Blindtext


\clearpage{}
\pagenumbering{Roman}
\setcounter{page}{7} %%% This page counter has to edited regarding the number of pages before the main part. If the listings ended with V, you have to set the counter to 6.


% Print the bibliography
\phantomsection
\markboth{Bibliography}{Bibliography}
\addcontentsline{toc}{chapter}{Bibliography}
\chapter*{Bibliography}

\printbibliography[notkeyword=Online, heading=literature] % List all the keywords not to be printed
\begingroup
	\raggedright
	\sloppy
	\printbibliography[keyword=Online, heading=internet] % List the keyword for this bibliography section
\endgroup


% Index ------------------------------------------------------------------------------------------------
%   Uncomment this line for creating an index.
% ------------------------------------------------------------------------------------------------------
%\printindex


% Appendix -----------------------------------------------------------------------------------------------
% The conents of the appendix are included like the chapters.
% See file "Appendix.tex"
% ------------------------------------------------------------------------------------------------------
\begin{appendix}
    \clearpage{}
    \pagenumbering{roman}
    \chapter{Appendix}
    \label{sec:Appendix}
    % Margin of the enumerations in the table
    \setdefaultleftmargin{1em}{}{}{}{}{}
    % !TEX encoding = UTF-8 Unicode
% !TEX root =  Paper.tex

\chapter{CD Attached}
\label{cha:CDAttached}

\begin{enumerate}
	\item The thesis as PDF
	\item All online resources as PDF
	\item Source code
	\item \ldots
\end{enumerate}
\end{appendix}

\end{document}
