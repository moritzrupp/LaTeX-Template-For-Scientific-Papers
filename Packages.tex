% !TEX encoding = UTF-8 Unicode
% !TEX root =  Paper.tex

% Anpassung des Seitenlayouts ---------------------------------------------------
%   siehe Seitenstil.tex
% -----------------------------------------------------------------------------------------
\usepackage[
    automark, % Kapitelangaben in Kopfzeile automatisch erstellen
    headsepline, % Trennlinie unter Kopfzeile
    ilines % Trennlinie linksbündig ausrichten
]{scrpage2}


% Anpassung an Landessprache -------------------------------------------------
\usepackage[UKenglish]{babel}


% Umlaute ------------------------------------------------------------------------------
%   Umlaute/Sonderzeichen wie äüöß direkt im Quelltext verwenden (CodePage).
%   Erlaubt automatische Trennung von Worten mit Umlauten.
%   Für Umlaute siehe Hauptdokument Zeile 41
% -----------------------------------------------------------------------------------------
\usepackage{textcomp} % Euro-Zeichen etc.


% Schrift --------------------------------------------------------------------------------
\usepackage{lmodern} % bessere Fonts
\usepackage{relsize} % Schriftgröße relativ festlegen


% Bessere Unterstreichungen ---------------------------------------------
\usepackage[normalem]{ulem}


% Grafiken -----------------------------------------------------------------------------
% Einbinden von JPG-Grafiken ermöglichen
\usepackage[dvips,final]{graphicx}
% hier liegen die Bilder des Dokuments
\graphicspath{{Images/}}
\usepackage{picture}


% Befehle aus AMSTeX für mathematische Symbole z.B. \boldsymbol \mathbb
\usepackage{amsmath,amsfonts}


% Eurozeichen benutzen ------------------------------------------------------------
\usepackage{eurosym}


% für Index-Ausgabe mit \printindex -----------------------------------------------
\usepackage{makeidx}


% Einfache Definition der Zeilenabstände und Seitenränder etc. ------------
\usepackage{setspace}
\usepackage{geometry}


% Symbolverzeichnis ------------------------------------------------------------------
%
%   Symbolverzeichnisse bequem erstellen. Beruht auf MakeIndex:
%     makeindex.exe %Name%.nlo -s nomencl.ist -o %Name%.nls
%   erzeugt dann das Verzeichnis unter WIndows. Dieser Befehl kann z.B. im TeXnicCenter
%   als Postprozessor eingetragen werden, damit er nicht ständig manuell
%   ausgeführt werden muss.
%
%   Unter Unix bitte einfach das mitgelieferte Shellskript "makemyindex" nutzen.
%
%   Die Definitionen werden im Fließtext mit den Befehlen:
%               - \Fachbegriff
%               - \FachbegriffSpezial
%               - \FachbegriffSpezialB
%    erzeugt oder separat in die Datei "Glossar.tex" eingetragen.
% -------------------------------------------------------------------------------------------
%\usepackage[intoc]{nomencl}
%\let\abbrev\nomenclature
%\renewcommand{\nomname}{Abkürzungsverzeichnis und Glossar}
%\setlength{\nomlabelwidth}{.25\hsize}
%\renewcommand{\nomlabel}[1]{#1 \dotfill}
%\setlength{\nomitemsep}{-\parsep}
\usepackage[
	printonlyused,
	footnote
]{acronym}

\usepackage[
	nonumberlist, %keine Seitenzahlen anzeigen
	%acronym,      %ein Abkürzungsverzeichnis erstellen
	%section,     %im Inhaltsverzeichnis auf section-Ebene erscheinen
	toc,          %Einträge im Inhaltsverzeichnis
]{glossaries}

% zum Umfließen von Bildern ---------------------------------------------------------
\usepackage[vflt]{floatflt}
\usepackage{subfigure}

% zum Einbinden von Programmcode -----------------------------------------------
\usepackage{listings}
\usepackage{xcolor} 
\definecolor{hellgelb}{rgb}{1,1,0.9}
\definecolor{colKeys}{rgb}{0.8,0,0.5}
\definecolor{colIdentifier}{rgb}{0.6,0,0.3}
\definecolor{colComments}{rgb}{0,0.5,0}
\definecolor{colString}{rgb}{0,0,1}

\lstset{
    float=htbp,
    basicstyle=\ttfamily\color{black}\small\smaller,
    identifierstyle=,%\color{colIdentifier},
    keywordstyle=\color{colKeys}\bfseries,
    stringstyle=\color{colString},
    commentstyle=\color{colComments},
    columns=flexible,
    tabsize=4,
    frame=single,
    extendedchars=true,
    showspaces=false,
    showstringspaces=false,
    numbers=left,
    numberstyle=\tiny,
    breaklines=true,
    backgroundcolor=\color{hellgelb},
    breakautoindent=true,
    escapeinside={(*}{*)},
    literate={Ö}{{\"O}}1 {Ä}{{\"A}}1 {Ü}{{\"U}}1 {ß}{{\ss}}2 {ü}{{\"u}}1 {ä}{{\"a}}1 {ö}{{\"o}}1 {µ}{\textmu}1
 }

% --------------------------------------------------------------------------------------------
% 
% Eigene Definitionen für Quelltext-Stile
%
% Define CSS:
\lstdefinelanguage{CSS}{
    morestring=[b]',
    morestring=[b]",
    comment=[l]{/*}{*/},
    sensitive=false,
    morekeywords={accelerator,adjust,after,align,attachment,azimuth,background,before,behavior,binding,border,bottom,bottomright,bottomleft,break,caption-side,char,clear,clip,color,colors,cue,cursor,collapse,decoration,direction,display,elevation,empty-cells,family,filter,float,flow,focus,font,grid,height,image,increment,indent,input,inside,ime-mode,include-source,justify,last,layer,left,letter,line,list,margin,marker,marks,max,min,mode,modify,moz,offsett,opacity,orphans,outline,overflow,overflow-X,overflow-Y,overhang,position,position-x,position-y,padding,page,pause,pitch,play-during,position,quotes,radius,range,repeat,replace,reset,richness,right,ruby,set-link-source,select,shadow,size,spacing,speak,speak-header,speak-numeral,speak-punctuation,speech-rate,stress,stretch,style,table-layout,text,transform,text-autospace,text-kashida-space,top,topleft,topright,type,underline,unicode-bidi,use-link-source,user,variant,vertical,visibility,voice-family,volume,white-space,weight,widows,width,word,wrap,word-wrap,writing-mode,z-index,zoom}
}


% Define JavaScript:
\lstdefinelanguage{JavaScript}{
  keywords={typeof, new, true, false, catch, function, return, null, catch, switch, var, if, in, while, do, else, case, break},
  keywordstyle=\color{colKeys}\bfseries,
  ndkeywords={class, export, boolean, throw, implements, import, this},
  ndkeywordstyle=\color{darkgray}\bfseries,
  sensitive=false,
  comment=[l]{//},
  morecomment=[s]{/*}{*/},
  morestring=[b]',
  morestring=[b]"
}

% Define TypoScript:
\lstdefinelanguage{TypoScript}{
  keywords=[1]{PAGE, HTML, TEXT, COA, COA\_INT, FILE, IMAGE, IMG\_RESOURCE, CLEARGIF, CONTENT, RECORDS, HMENU, CTABLE, OTABLE, COLUMNS, HRULER, IMGTEXT, CASE, LOAD\_REGISTER, RESTORE\_REGISTER, FORM, SEARCHRESULT, USER, USER\_INT, TEMPLATE, FLUIDTEMPLATE, MULTIMEDIA, SVG, EDITPANEL, GIFBUILDER, GMENU, TMENU, TMENUITEM, IMGMENU, IMGMENUITEM, JSMENU, JSMENUITEM, BOX},
  keywordstyle=[1]{\color{blue}\bfseries},
  keywords=[2]{plugin, view, page, file, text, config},
  keywordstyle=[2]{\color{blue}\bfseries},
  keywords=[3]{EXT},
  keywordstyle=[3]{\color{blue}\bfseries},
  sensitive=true,
  comment=[l]{\#\ }
}

% PHP5 dialect 
\lstdefinelanguage[5]{PHP}[]{PHP} {
  morekeywords={
  %--- class and exceptions keywords 
  class,static,private,public,abstract,interface,const,function,require_once,final,new,extends,implements,
  %--- additional array functions 
  array_combine,array_diff_uassoc,array_udiff,array_udiff_assoc,% 
  array_udiff_uassoc,array_walk_recursive,array_uintersect_assoc,% 
  array_usintersect_uassoc,array_uintersect,% 
  %--- string functions 
  str_split, strpbrk,substr_compare 
  %--- Date and time functions 
  idate,date_sunset,date_sunrise,time_nanosleep,
  %--- return
  return
  } 
} 


% Fluid dialect 
\lstdefinelanguage[Fluid]{HTML}[]{HTML} {
  morekeywords={if, for, each, as, condition, controller, arguments, image, link, action, class} 
} 


% URL verlinken, lange URLs umbrechen etc. -------------------------------------
\usepackage{url}


% natbib einbinden -----------------------------------------------------------------------
\usepackage[round, numbers, sort, authoryear]{natbib}


% PDF-Optionen -------------------------------------------------------------------------
\usepackage[
    bookmarks,
    bookmarksopen=true,
    colorlinks=true,
% diese Farbdefinitionen zeichnen Links im PDF farblich aus
    linkcolor=red, % einfache interne Verknüpfungen
    anchorcolor=black,% Ankertext
    citecolor=blue, % Verweise auf Literaturverzeichniseinträge im Text
    filecolor=magenta, % Verknüpfungen, die lokale Dateien öffnen
    menucolor=red, % Acrobat-Menüpunkte
    urlcolor=cyan, 
%
% diese Farbdefinitionen sollten für den Druck verwendet werden (alles schwarz):
%
    %linkcolor=black, % einfache interne Verknüpfungen
    %anchorcolor=black, % Ankertext
    %citecolor=black, % Verweise auf Literaturverzeichniseinträge im Text
    %filecolor=black, % Verknüpfungen, die lokale Dateien öffnen
    %menucolor=black, % Acrobat-Menüpunkte
    %urlcolor=black, 
%
    backref,
    plainpages=false, % zur korrekten Erstellung der Bookmarks
    pdfpagelabels, % zur korrekten Erstellung der Bookmarks
    hypertexnames=false, % zur korrekten Erstellung der Bookmarks
    linktocpage % Seitenzahlen anstatt Text im Inhaltsverzeichnis verlinken
]{hyperref}
% Befehle, die Umlaute ausgeben, führen zu Fehlern, wenn sie hyperref als Optionen übergeben werden
\hypersetup{
    pdftitle={\titel \untertitel},
    pdfauthor={\autor},
    pdfcreator={\creator},
    pdfsubject={\titel \untertitel},
    pdfkeywords={\keywords},
}

% Paket zum sauberen Einbauen von externen PDF-Dateien -----------------
\usepackage[final]{pdfpages}


% fortlaufendes Durchnummerieren der Fußnoten -------------------------------
\usepackage{chngcntr}


% schönere Tabellen --------------------------------------------------------------------
\usepackage{tabularx}
\usepackage{multirow}


% für lange Tabellen ---------------------------------------------------------------------
\usepackage{longtable}
\usepackage{ltxtable}
\usepackage{filecontents}
\usepackage{array}
\usepackage{ragged2e}
\usepackage{lscape}


% Rotation von Elementen -------------------------------------------------------
\usepackage{rotating}


% Formatierung von Listen ändern --------------------------------------------------
\usepackage{paralist}
\usepackage{enumitem}


% bei der Definition eigener Befehle benötigt -------------------------------------
\usepackage{ifthen}
\usepackage{forloop}


% definiert u.a. die Befehle \todo und \listoftodos --------------------------------
\usepackage[english]{todonotes}


% sorgt dafür, dass Leerzeichen hinter parameterlosen Makros nicht als Makroendezeichen interpretiert werden
\usepackage{xspace}
