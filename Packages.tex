% !TEX encoding = UTF-8 Unicode
% !TEX root =  Arbeit.tex

% Anpassung des Seitenlayouts ---------------------------------------------------
%   siehe Seitenstil.tex
% -----------------------------------------------------------------------------------------
\usepackage[
    automark, % Kapitelangaben in Kopfzeile automatisch erstellen
    headsepline, % Trennlinie unter Kopfzeile
    ilines % Trennlinie linksbündig ausrichten
]{scrpage2}

% Verhindert bestimmte Warnungen
\usepackage{scrhack}

% Anpassung an Landessprache -------------------------------------------------
\usepackage[ngerman]{babel}


% Umlaute ------------------------------------------------------------------------------
%   Umlaute/Sonderzeichen wie äüöß direkt im Quelltext verwenden (CodePage).
%   Erlaubt automatische Trennung von Worten mit Umlauten.
%   Für Umlaute siehe Hauptdokument Zeile 41
% -----------------------------------------------------------------------------------------
\usepackage{textcomp} % Euro-Zeichen etc.


% Schrift --------------------------------------------------------------------------------
\usepackage{lmodern} % bessere Fonts
\usepackage{relsize} % Schriftgröße relativ festlegen


% Bessere Unterstreichungen ---------------------------------------------
\usepackage[normalem]{ulem}


% Grafiken -----------------------------------------------------------------------------
% Einbinden von JPG-Grafiken ermöglichen
\usepackage[dvips,final]{graphicx}
% hier liegen die Bilder des Dokuments
\graphicspath{{Bilder/}}
\usepackage{picture}


% Befehle aus AMSTeX für mathematische Symbole z.B. \boldsymbol \mathbb
\usepackage{amsmath,amsfonts}


% Eurozeichen benutzen ------------------------------------------------------------
\usepackage{eurosym}


% für Index-Ausgabe mit \printindex -----------------------------------------------
\usepackage{makeidx}


% Einfache Definition der Zeilenabstände und Seitenränder etc. ------------
\usepackage{setspace}
\usepackage{geometry}

% zum Umfließen von Bildern ---------------------------------------------------------
\usepackage[vflt]{floatflt}
\usepackage{subfigure}
\usepackage{tocbasic}

% zum Einbinden von Programmcode -----------------------------------------------
\usepackage{listings}
\usepackage{xcolor} 
\definecolor{hellgelb}{rgb}{1,1,0.9}
\definecolor{colKeys}{rgb}{0.8,0,0.5}
\definecolor{colIdentifier}{rgb}{0.6,0,0.3}
\definecolor{colComments}{rgb}{0,0.5,0}
\definecolor{colString}{rgb}{0,0,1}

\lstset{
    float=htbp,
    basicstyle=\ttfamily\color{black}\small\smaller,
    identifierstyle=,%\color{colIdentifier},
    keywordstyle=\color{colKeys}\bfseries,
    stringstyle=\color{colString},
    commentstyle=\color{colComments},
    columns=flexible,
    tabsize=4,
    frame=single,
    extendedchars=true,
    showspaces=false,
    showstringspaces=false,
    numbers=left,
    numberstyle=\tiny,
    breaklines=true,
    backgroundcolor=\color{hellgelb},
    breakautoindent=true,
    escapeinside={(*}{*)},
    literate={Ö}{{\"O}}1 {Ä}{{\"A}}1 {Ü}{{\"U}}1 {ß}{{\ss}}2 {ü}{{\"u}}1 {ä}{{\"a}}1 {ö}{{\"o}}1 {µ}{\textmu}1
 }

% --------------------------------------------------------------------------------------------
% 
% Eigene Definitionen für Quelltext-Stile
%
% Beispieldefinition von CSS:
\lstdefinelanguage{CSS}{
    morestring=[b]',
    morestring=[b]",
    comment=[l]{/*}{*/},
    sensitive=false,
    morekeywords={accelerator,adjust,after,align,attachment,azimuth,background,before,behavior,binding,border,bottom,bottomright,bottomleft,break,caption-side,char,clear,clip,color,colors,cue,cursor,collapse,decoration,direction,display,elevation,empty-cells,family,filter,float,flow,focus,font,grid,height,image,increment,indent,input,inside,ime-mode,include-source,justify,last,layer,left,letter,line,list,margin,marker,marks,max,min,mode,modify,moz,offsett,opacity,orphans,outline,overflow,overflow-X,overflow-Y,overhang,position,position-x,position-y,padding,page,pause,pitch,play-during,position,quotes,radius,range,repeat,replace,reset,richness,right,ruby,set-link-source,select,shadow,size,spacing,speak,speak-header,speak-numeral,speak-punctuation,speech-rate,stress,stretch,style,table-layout,text,transform,text-autospace,text-kashida-space,top,topleft,topright,type,underline,unicode-bidi,use-link-source,user,variant,vertical,visibility,voice-family,volume,white-space,weight,widows,width,word,wrap,word-wrap,writing-mode,z-index,zoom}
}


% URL verlinken, lange URLs umbrechen etc. -------------------------------------
\usepackage{url}


% biblatex einbinden -----------------------------------------------------------------------
\usepackage[
	citestyle=my-authoryear-ibid,
	bibstyle=authoryear,
	backend=biber,
	babel=hyphen,
	urldate=comp,
	dateabbrev=false,
	maxcitenames=1
]{biblatex}
\usepackage[babel,german=quotes]{csquotes}


% PDF-Optionen -------------------------------------------------------------------------
\usepackage[
    bookmarks,
    bookmarksopen=true,
    colorlinks=true,
% diese Farbdefinitionen zeichnen Links im PDF farblich aus
    linkcolor=red, % einfache interne Verknüpfungen
    anchorcolor=black,% Ankertext
    citecolor=blue, % Verweise auf Literaturverzeichniseinträge im Text
    filecolor=magenta, % Verknüpfungen, die lokale Dateien öffnen
    menucolor=red, % Acrobat-Menüpunkte
    urlcolor=cyan, 
%
% diese Farbdefinitionen sollten für den Druck verwendet werden (alles schwarz):
%
    %linkcolor=black, % einfache interne Verknüpfungen
    %anchorcolor=black, % Ankertext
    %citecolor=black, % Verweise auf Literaturverzeichniseinträge im Text
    %filecolor=black, % Verknüpfungen, die lokale Dateien öffnen
    %menucolor=black, % Acrobat-Menüpunkte
    %urlcolor=black, 
%
    plainpages=false, % zur korrekten Erstellung der Bookmarks
    pdfpagelabels, % zur korrekten Erstellung der Bookmarks
    hypertexnames=false, % zur korrekten Erstellung der Bookmarks
    linktocpage % Seitenzahlen anstatt Text im Inhaltsverzeichnis verlinken
]{hyperref}
% Befehle, die Umlaute ausgeben, führen zu Fehlern, wenn sie hyperref als Optionen übergeben werden
\hypersetup{
    pdftitle={\titel \untertitel},
    pdfauthor={\autor},
    pdfcreator={\creator},
    pdfsubject={\titel \untertitel},
    pdfkeywords={\keywords},
}

% Glossar und Abkürzungsverzeichnis -------------------------
\usepackage[
	%xindy,
	nonumberlist,	% keine Seitenzahlen anzeigen
	acronym,		% ein Abkürzungsverzeichnis erstellen
	toc,			% Einträge im Inhaltsverzeichnis
	shortcuts		% acronym shortcuts
]{glossaries}

% Paket zum sauberen Einbauen von externen PDF-Dateien -----------------
\usepackage[final]{pdfpages}


% fortlaufendes Durchnummerieren der Fußnoten -------------------------------
\usepackage{chngcntr}


% schönere Tabellen --------------------------------------------------------------------
\usepackage{tabularx}
\usepackage{multirow}


% für lange Tabellen ---------------------------------------------------------------------
\usepackage{longtable}
\usepackage{ltxtable}
\usepackage{filecontents}
\usepackage{array}
\usepackage{ragged2e}
\usepackage{lscape}


% Rotation von Elementen -------------------------------------------------------
\usepackage{rotating}


% Formatierung von Listen ändern --------------------------------------------------
\usepackage{paralist}
\usepackage{enumitem}


% bei der Definition eigener Befehle benötigt -------------------------------------
\usepackage{ifthen}
\usepackage{forloop}


% definiert u.a. die Befehle \todo und \listoftodos --------------------------------
\usepackage[ngerman]{todonotes}


% sorgt dafür, dass Leerzeichen hinter parameterlosen Makros nicht als Makroendezeichen interpretiert werden
\usepackage{xspace}

% Für fette Bibliographieeinträge
\usepackage{xpatch}

% Für mehrspaltige und mehrzeilige Tabellen
\usepackage{multicol}
\usepackage{multirow}

% Bessere Silbentrennung etc.
\usepackage{microtype}

% Blindtext im Inhaltsbereich
\usepackage{blindtext}