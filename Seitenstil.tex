% !TEX encoding = UTF-8 Unicode
% !TEX root =  Arbeit.tex

% Zeilenabstand 1,5 Zeilen ---------------------------------------------------------------------
\onehalfspacing{}


% Seitenränder ------------------------------------------------------------------------------------
\setlength{\topskip}{\ht\strutbox} % behebt Warnung von geometry
\geometry{paper=a4paper,left=35mm,right=35mm,top=35mm}


% Kopf- und Fußzeilen --------------------------------------------------------------------------
\pagestyle{scrheadings}

% Kopf- und Fußzeile auch auf Kapitelanfangsseiten
\renewcommand*{\chapterpagestyle}{scrheadings} 

% Schriftform der Kopfzeile
\renewcommand{\headfont}{\normalfont}


% Kopfzeile für einseitiges Dokument:
\ihead{
	\textit{\headmark}
}
\chead{}
\rohead{\includegraphics[scale=0.06]{\logo}}
\setlength{\headheight}{21mm} % Höhe der Kopfzeile

% Kopfzeile über den Text hinaus verbreitern
%\setheadwidth[0pt]{textwithmarginpar} 
\setheadsepline[text]{0.4pt} % Trennlinie unter Kopfzeile


%% Kopfzeile für Zweiseitiges Dokument:
%\ihead{\textit{\leftmark}}
%\chead{}
%\ohead{}
%\lehead{\includegraphics[scale=0.25]{\logo}}
%\rohead{\includegraphics[scale=0.25]{\logo}}
%\setheadsepline[text]{0.4pt} % Trennlinie unter Kopfzeile


% Fußzeile
%\cfoot{-- ENTWURF -- \\[4ex]} % Einkommentieren, während des Entwurfstatus
\cfoot{} % Auskommentieren, während des Entwurfstatus
\ofoot{\pagemark \\[4ex]}


% sonstige typographische Einstellungen ---------------------------------------------------
% erzeugt ein wenig mehr Platz hinter einem Punkt
\frenchspacing{}

% Schusterjungen und Hurenkinder vermeiden
\clubpenalty = 10000
\widowpenalty = 10000 
\displaywidowpenalty = 10000

% Quellcode-Ausgabe formatieren
\lstset{numbers=left, numberstyle=\tiny, numbersep=5pt, breaklines=true}
\lstset{emph={square}, emphstyle=\color{red}, emph={[2]root,base}, emphstyle={[2]\color{blue}}}

% Fußnoten fortlaufend durchnummerieren
\counterwithout{footnote}{chapter}

% Mehr Platz zwischen den Zellen von Tabellen (Wert lässt sich anpassen)
\renewcommand{\arraystretch}{\TableCellPadding}

% Seitenstil für die Bibliographie
\setlength\bibitemsep{2\itemsep} % Abstand zwischen den Einträgen
\setlength{\bibhang}{0.2cm}

% Fette Bibliographieeinträge
\xpretobibmacro{author}{\mkbibbold\bgroup}{}{}
\xapptobibmacro{author}{\egroup}{}{}
\xpretobibmacro{bbx:editor}{\mkbibbold\bgroup}{}{}
\xapptobibmacro{bbx:editor}{\egroup}{}{}
\renewcommand*{\labelnamepunct}{\mkbibbold{\addcolon\space}}